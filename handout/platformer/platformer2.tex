\documentclass{../cs195u}

\usepackage{url}

\asgn{Platformer 2}
\due{March 1, 2021}

\begin{document}
 \section*{Introduction}

Now that your engine supports movement around an arbitrary environment mesh, let's optimize how it handles collisions!


 \section*{Directions}
  
For this checkpoint, you will implement a spatial organization data structure. You may choose which data structure you implement (BVH, Uniform Grid, Hierarchical Grid, Octree, K-D Tree, BSP Tree). Then, you will use this data structure to speed up your collision checking between the player and the static environment mesh. After that, you will use a data structure to speed up your collision checking between dynamic objects. You may use different data structures to handle static and dynamic objects.


  \section*{Design Check}
  \begin{itemize}
   \item What data structure(s) will you implement? How will your data structure(s) handle static and dynamic objects?
   \item What will be the member variables of the nodes of your data structure?
   \item How will you build your data structure?
   \item How will you traverse your data structure?
   \item Will you need to update your data structure as the game state changes?
  \end{itemize}

  \section*{Engine Requirements}
  \begin{itemize}
    \item Spatial organization data structure that handles collisions with static objects
    \item Spatial organization data structure that handles collisions between dynamic objects 
 \end{itemize}
 
 \section*{Game Requirements}
  \begin{itemize}
    \item Screen populated with enough dynamic objects (e.g. collidable cylinders) to demonstrate benefit of spatial organization data structure
    \item Screen with a large enough environment to demonstrate benefit of spatial organization data structure
   \end{itemize}

 \section*{Handing In}
Hand in the entire directory tree for your project, including both your engine and game code. You must also include a README file that describes how to verify each requirement, and an INSTRUCTIONS file that describes how to play your game. To hand in, run \texttt{cs195u\_handin platformer2} from the top level directory of your project (which should be where your Qt pro file is). \textbf{Please do not hand in the build files from your project.}

\end{document}
