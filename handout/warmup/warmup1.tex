\documentclass{../cs195u}

\usepackage{url}

\asgn{Warmup 1}
\due{Feb. 1, 2021}

\begin{document}
 \section*{Introduction}

Welcome to CS1950U! In this assignment you’ll be creating the basic framework of the game
engine you will be developing for the rest of the semester. It will introduce first and third-person
movement in a 3D world, basic application and game world organization, and some important
graphics concepts. By the end of these two weeks, you’ll also have a simple 3D game built on top
of your engine!

All assignments in this course will follow a similar format of requiring both engine features and a
game built on the engine.

  Warmup 1 will get you used to working in 3D space. Though there won't always be a ton of requirements related to gameplay, this is ultimately a class about making games, so have fun with these assignments, and feel free to build more than is required! What is listed here is just the bare minimum for getting a complete on a checkpoint.


%To make sure you fulfill the requirements for a given checkpoint, you should copy the rubric for the
%checkpoint into your README. The rubric will contain the same requirements as this document,
%but should be easier to copy directly into your README. Rubrics can be found at \path{/course/cs195u/rubrics/<asgn>/grade.txt}.

%Before moving on, we recommend reading the Course Missive on the Docs and Resources page to
%make sure you understand the grading policy, and structure of the course.
%If you want to see an example of what Warmup might look like, feel free to run \texttt{cs195u\_demo warmup\{1,2\}} on any department machine.


 \section*{Support Files and IDE}
  \subsection*{Getting Started}
 If you haven’t yet, read the first few pages of the CS1950U Setup Guide on the Docs
page. It contains instructions for setting up a work environment on your personal computer as well
as more detailed descriptions of the support code.
  
To get the support code for this and all future assignments, copy the contents of \path{/course/cs195u/asgn/engine} to your project directory for Warmup (probably something like \path{~/course/cs195u/engine}). This should give you a basic Qt project including a Qt pro file that configures and helps build your project, as well as directories containing some starter code and resources. This code should compile and run right away, give you a black window and a framerate counter.

You’ll use the Qt Creator IDE for this course. Open Qt Creator (run the \path{cs195u_qtcreator} command if you are on a department machine) and open \path{cs195u_engine.pro} to load the project. Files ending in .pro are text files containing the project configuration (a list of sources, compiler flags, and platform-specific build commands). Here are several shortcuts you can use in Qt Creator:


  \begin{itemize}
   \item Ctrl+R: Build and run your project (standard out appears in the Application Output pane)
   \item Ctrl+K: Quickly open any file in the project by name (in addition to any class or function)
   \item Ctrl+Click: Jump to the definition of any symbol (variable, function, macro, etc.)
   \item F4: Switch between *.h and *.cpp files with the same name
  \end{itemize}

  \subsection*{Support Code}
  A major part of game engine development is being able to design large and complex software systems. For this reason, the support code for CS1950U is minimal. You will need to implement the majority of each project from scratch, so make sure to allocate time for design. 

  That being said, we realize you have limited time and want you to focus on what's interesting, so we have provided a few support files to get your started:

  \begin{itemize}
   \item \path{view.{h,cpp}}: Defines a View widget extending QGLWidget. This is a starting point for your game engine; it sets up a full screen window with mouse capture and a variable-update game loop. Every update of the game loop calls tick() to handle game updates and triggers paintGL() to redraw the view. You will want to fill in these methods when implementing Warmup.
   \item \path{mainwindow.{h,cpp,ui}}: Initializes a main window containing a View widget. You should not need to modify these files.
   \item \path{main.cpp}: Starts the program.
   \item \path{util/CommonIncludes.h}: Contains \texttt{include} statements for universally needed classes and libraries, such as glm. You may add whatever you want to it, however it will be included in many, many files, so try to keep it as small as possible.
\item \path{engine/graphics/Graphics.h}: Contains an implementation of a graphics object, which
includes various functions for drawing, and OpenGL state management. It also manages
graphics resources such as Textures, Shaders, Shapes, Materials, Fonts, and Framebuffers.
\item \path{engine/graphics/Camera.h}: A default implementation of a Camera object, which describes
a view on a 3D world.
\item \path{engine/graphics/*}: Contains other graphics helper classes which you’ll hopefully find useful!
  \end{itemize}

  Note: when you create new folders, you may want to add them to INCLUDEPATH and DEPENDPATH in warmup.pro so you can \#include files inside them directly.

  \subsection*{Resources}
  In addition to the support code, we may also provide resources such as textures and models. In this assignment, the only resource we provide is a grass texture entitled \path{grass.png} (from \url{opengameart.org}). Feel free to find other textures or models while doing your projects, but keep in mind that you cannot use any associated code.
  \newpage

  \subsection*{Design Check}
  \begin{itemize}
   \item List the steps involved with setting up a first person camera. How will you get or compute the parameters you need?
   \item How will you define an application? What about a screen?
   \item In broad terms, describe the steps and OpenGL calls necessary to render the floor.
   \item How will you implement gravity, the floor, and jumping?
  \end{itemize}
  
\subsection*{Basic Requirements}
Basic requirements describe what you need in order to have a basic, playable demo for a checkpoint.
For Warmup 1, they are as follows:
 \begin{itemize}
\item Handin never crashes
%\item Handin runs on department machines at 20+ FPS
\item Your game renders a quad in 3D
\item Moving the mouse pans the camera
\item Pressing certain keys moves the camera
 \end{itemize}

  \subsection*{Engine Requirements}
  The following will all be a part of your ``Common'' engine. For now, you don't need to know what this means. Just know that the following features should be applicable to almost any game, as a result, should be logically separated from, but utilized by, your game code:
  \begin{itemize}
    \item Virtual Application class representing a whole game which supports:
    \begin{itemize}
     	\item Timed updates (tick)
     	\item Render events (draw)
     	\item Input events (mouse and keyboard)
     	\item Window size updates (resize)
	\item Adding and removing screens
	\item Switching screens
    \end{itemize}
    \item Virtual Screen class representing a logical subscreen of a game which (minimally) supports:
    \begin{itemize}
     \item Timed updates (tick)
     \item Render events (draw)
     \item Input events (mouse and keyboard)
     \item Window size updates (resize)
    \end{itemize}
\item Your engine uses the provided Camera object, or you’ve built your own Camera object
\item Your engine uses the provided Graphics object, or you’ve built your own Graphics object
   \end{itemize}
 
  \subsection*{Game Requirements}
  For this week, you will not implement any real gameplay. Your handin should allow the player to walk around a world using mouse and keyboard inputs to change the camera. The player will be able to jump and not fall through a textured floor, but there will be no ``point'' to the game.
 
  Your game must fulfill the following set of requirements:
  \begin{itemize}
   \item The player must be able to move using standard first person camera controls, meaning:
   \begin{itemize}
    \item Horizontal mouse movements change the yaw of the camera
    \item Vertical mouse movements change the pitch of the camera
    \item Standard WASD keyboard controls (W moves forwards, S moves backwards, A strafes left, D strafes right) change the eye of the camera
   \end{itemize}
   \item The player must never fall through the ground at $y = 0$
   \item The player must be able to jump off the ground using the spacebar or a mouse button
   \item The player can only jump when on the ground
   \item Gravity must act downwards on the player
   \item The ground must consist of planar geometry with a tiled grass texture. This means the floor is a series of 1x1 quads each with the same texture, not a single quad with a stretched texture
   \item The game must have \textbf{at least two} screens, one of which requires player input to get to the other
  \end{itemize}

%Keep in mind that there are also global requirements that apply to every checkpoint. Be sure to confirm that you meet these as well!

 \section*{Handing In}
Hand in the entire directory tree for your project, including both your engine and game code. You must also include a README file that describes how to verify each requirement, and an INSTRUCTIONS file that describes how to play your game. To hand in, run \texttt{cs195u\_handin warmup1} from the top level directory of your project (which should be where your Qt pro file is). \textbf{Please do not hand in the build files from your project.}

\end{document}
