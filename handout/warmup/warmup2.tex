\documentclass{../cs195u}

\usepackage{url}

\asgn{Warmup 2}
\due{Feb. 8, 2021}

\begin{document}
 \section*{Introduction}
Warmup 2 will add world organization functionality to your game engine, as well as a basic collision system and a third-person camera. By the end of the week, you'll have your first fully playable game!

  \section*{Design Check}
  \begin{itemize}
   \item How will your camera support both first and third person?
   \item Describe the steps to determine if, and by how much, two cylinders are colliding.
   \item What are the steps involved with determining a collision? Which are engine-side, and which are game-side?
   \item What kind of gameplay will you implement?
  \end{itemize}

\section*{Basic Requirements}
 \begin{itemize}
\item Handin never crashes
\item Handin runs on department machines at 20+ FPS
\item All game logic is contained in a GameWorld class; Screens have no game logic
\item There is some non-trivial game logic (e.g., jumping, colliding with ground, etc.)
 \end{itemize}
  
\section*{Engine Requirements}
Like week 1, all code written this week will be a part of your ``Common'' engine:
\begin{itemize}
	\item World/System/GameObject/Component hierarchy
   	\begin{itemize}
    		\item GameWorld class representing representing the game world which supports:
    		\begin{itemize}
     			\item Timed updates (tick)
     			\item Render events (draw)
     			\item Adding and removing Systems
     			\item Adding and removing GameObjects
    		\end{itemize}

  		\item System class representing one type of functionality that your world supports, containing:
    		\begin{itemize}
    			 \item Timed updates (tick)
    			 \item Adding and removing GameObjects
    		\end{itemize}

		\item GameObject class representing one entity within your game, supporting:
		 \begin{itemize}
			\item Adding and removing Components
		\end{itemize}

  		 \item Component class representing the smallest unit of functionality in a single entity, supporting:
 		\begin{itemize}
       			\item Timed updates (tick)
		\end{itemize}
	\end{itemize}

	\item Engine supports an input map
	\item Engine has a TickSystem
	\item Engine has a DrawSystem
	\item Engine supports a third-person camera

   	\item CollisionSystem that can handle Cylinder-Cylinder collisions
  	 \begin{itemize}
   		 \item Detection - determine that two cylinders are overlapping
   		 \item Resolution - translate the cylinders out of detection using the MTV
   		 \item Response - dispatches collision callback
  	 \end{itemize}
\end{itemize}
 
  \section*{Game Requirements}
  Now that you have some basic collisions, you can complete your first game! We're intentionally vague in the following requirements to encourage you to think outside the box: what can you do with just cylinder-cylinder collisions and a flat floor? We're looking forward to seeing the results.
 
  Your game must fulfill the following set of requirements:
  \begin{itemize}
   \item The player should be able to jump (but only when on the ground, or under other specific conditions)
 \item The player shouldn't fall through the ground
 \item The player should fall under the effects of gravity
 \item Camera and player movement should be continuous
 \item Cylinder-cylinder collisions are used somewhare
 \item There exists an NPC
 \item The NPC has some basic AI
 \item The game has some non-arbitrary win or loss condition
 \item The game must be resettable
 \item The game cannot enter an unwinnable or unlosable scenario (or resets upon doing so)
  \end{itemize}

 \section*{Handing In}
Hand in the entire directory tree for your project, including both your engine and game code. You must also include a README file that describes how to verify each requirement, and an INSTRUCTIONS file that describes how to play your game. To hand in, run \texttt{cs195u\_handin warmup2} from the top level directory of your project (which should be where your Qt pro file is). \textbf{Please do not hand in the build files from your project.}

\end{document}
