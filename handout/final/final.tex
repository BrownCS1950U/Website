\documentclass{../cs195u}

\usepackage{url}

\asgn{Final}
\due{April 18, 2021}

\begin{document}
  
\section*{Introduction}
For your final project in Topics in 3D Game Engine Development, you will be creating your very own 3D game! You will spend one week doing research on game and engine features and creating a proposal for your project. Then, you will spend 4 weeks implementing your project.
  
 \section*{Final Plan}
 \subsection*{Due March 22}
  Each group must complete a project proposal. This is a document containing the following information:
  
  \begin{itemize}
  	\item Group member's names
	\item Engine or game features you plan to implement - Refer to the Engine/Game Features section of this handout for possible features. You should pick out the features you think are essential to your game. 
	\item Game concept - Briefly describe the game and explain how your features fit into the game. 
  \end{itemize}

In your proposal, it should be clear how your project will be an appropriate amount of work for your group size. Please be as specific as possible. Keep in mind that if you plan on implementing easy engine features, then you will be expected to make a polished game. If you plan on implementing difficult engine features, then your project will not be expected to be as polished. Do some research to find features that you are excited to implement. Ask the TA staff for project ideas if you are not sure what would be an appropriate amount of work. If you do a reasonable amount of research, then a suitable project idea should come naturally. See the Engine Features section for more information on what separates "easy" and "difficult" engine features.

\textbf{If you are taking this course as a senior capstone}, you must work alone or with other students taking the capstone. Students taking the course as a capstone will be expected to implement difficult engine features and make a polished game. \textbf{Students taking the capstone should get their project proposals approved before March 22 so that they can start early.}

\section*{Project Check-Ins}

We will have project check-ins every week during class time. You are encouraged to come to class (or hours) to get feedback on your progress!
 
\section*{Final Handin}
\subsection*{Due April 18}

Hand in your project code before 11:59pm on April 18. In addition, \textbf{hand in a text file containing a link to a video of your game on Google Drive or Dropbox}. Also, make sure to include a detailed README and INSTRUCTIONS.  

Final project presentations will be on April 19 at 9am.

\section*{Engine/Game Features}

Keep in mind that there are tutorials online providing implementations of some of these features. For instance, it is very easy to find code for shadow mapping or bump mapping. Most features with readily available implementations that can be referenced are considered "easy". You can get a small amount of credit for implementing easy features, but you will be expected to either implement more difficult features or create a very polished game as well. A "difficult" feature is anything that is not easily implemented by following a tutorial with code, or a feature that takes considerable effort to incorporate into a game engine, such as networking. 

Some "easy" features can be extended and become "difficult" features. For instance, there are lots of creative things you can do with particles, such as animating them with a texture atlas. There are many ways you can extend shadow maps to improve their look and performance.

Also note that game-side features, like original assets or portals, only make sense to pursue if you are planning on making a polished game.

This project is an opportunity to pursue what makes YOU interested in game engines. If you want to make a polished game, then go for it! If you want to make a lightmapper and then create a walking simulator with your light maps, then do that!
 
Here is a list of possible features:
 
 \begin{itemize}
  \item Networking 
  \item Graphics features
  	\begin{itemize} 
	\item Particles 
	\item Bump mapping, Parallax mapping
	\item Deferred shading
	\item Depth of field
	\item Toon shading, Silhouette edges
	\item Shadows
	\item Ambient Occlusion
	\item Real-Time Indirect Illumination
		\begin{itemize}
		\item Raytracing (Hybrid Rendering)
		\item Light Probes 
		\item Screen Space Directional Occlusion 
		\item Light Propagation Volumes
		\end{itemize}
	\end{itemize}
  \item Advanced animation features (Inverse kinematics, Dual quaternion skinning)
  \item Advanced AI 
  \item Polished UI toolkit
  \item Chunk streaming (for a large world)
  \item Occlusion culling
  \item Tools
  	\begin{itemize} 
	\item AI editor
	\item Embedded scripting
	\item Lightmapper
	\item Procedural content generation
	\item Level editor
	\end{itemize}
  \item Sound 
  \item Really cool original assets 
  \item Portals 
  \item Physics
  	\begin{itemize}
   	 \item Rigid body physics
	 \item Ragdoll physics
   	 \item Fluid simulation
	 \item Cloth simulation, clothing animation
	 \end{itemize}
  \end{itemize}



\end{document}
